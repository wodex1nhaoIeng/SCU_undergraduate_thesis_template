%--------------------------------------------------------------------
%导言区
% !Mode:: "TeX:UTF-8"

\documentclass[a4paper,UTF8, AutoFakeBold]{ctexrep}

\usepackage{scuthesis}				% 封面版式
\usepackage{amssymb}  				% 设置数学公式
\usepackage{amsmath} 				% 设置数学公式编号
\usepackage{amsthm}					% 定理

%\usepackage[all,cmtip]{xy}			% xy-pic	画交换图
\usepackage{tikz}			        % tikz      绘图宏包
\usepackage{float}	  				% float		为固定图片位置宏包
\usepackage{subfigure}				% subfigure 引入宏包来添加多张图片 
\usepackage{caption}				% caption   为更改图片命名的宏包
\usepackage{enumerate}				% enumerate 有序列表环境
\usepackage{ctex}

%\usepackage{boondox-cal}			% boondox-cal   数学花体
%\usepackage{bm}       				% bm 			希腊字母加粗(普通字母类同)

\theoremstyle{plain}
	\newtheorem{thm}{定理~}[chapter]
	\newtheorem{lem}[thm]{引理~}
	\newtheorem{prop}[thm]{命题~}
	\newtheorem{cor}[thm]{推论~}
\theoremstyle{definition}
	\newtheorem{defn}[thm]{定义~}
	\newtheorem{conj}[thm]{猜想~}
	\newtheorem{exmp}[thm]{例~}
	\newtheorem{ques}[thm]{问题~}
	\newtheorem{rem}[thm]{注~}

%	该命令指定公式编号的格式
\numberwithin{equation}{chapter}
\renewcommand{\theequation}{\thechapter.\roman{equation}}

% 请在此处添加或修改你想要的定理样式,以下为英文定理样式,若使用中文写作请注释以下部分,并改用上面的中文定理样式(注意,使用英文写作时,若完成该操作后仍有部分单词显示为中文,请参照ctex文档第6节的“文档汉化”部分自行调整):
% \theoremstyle{plain}
%   \newtheorem{thm}{Theorem}[chapter]
%   \newtheorem{lem}[thm]{Lemma}
%   \newtheorem{prop}[thm]{Proposition}
%   \newtheorem{cor}[thm]{Corollary}
% \theoremstyle{definition}
%   \newtheorem{defn}[thm]{Definition}
%   \newtheorem{conj}[thm]{Conjecture}
%   \newtheorem{exmp}[thm]{Example}
%   \newtheorem{ques}[thm]{Question}
%   \newtheorem{rem}[thm]{Remark}
% \ctexset{bibname = {References}}
% \ctexset{proofname = {Proof}}
% \ctexset{contentsname = {Contents}}

%------------------------------------------------
	%基本信息
	\title{四川大学本科毕业论文模板(非官方)四川大学本科毕业论文模板}
	\titleEng{SCU Undergraduate Thesis Template(unofficial)SCU Undergraduate Thesis Template}
	
	\author{孙悟空}
	\authorEng{Wukong Sun}
	\adviser{唐三藏}
	\adviserEng{Sanzang Tang}
	
	\college{数学学院}
	\collegeEng{School of Mathematics}
	\major{数学与应用数学}
	\majorEng{Mathematics and Applied Mathematics}
	
	\grade{20xx}
	\id{20xxxxxxxxxxx}
	\date{\today}
	
 			% 作者信息


%-----------------------------------------------------------------
%正文区
	\begin{document}
	\zihao{-4}
	
	% 封面+摘要
	\makecover

\begin{abstract}{\LaTeX; 本科; 毕业论文; 模板}
本模版是根据overleaf上的 2022 年的 \href{https://www.overleaf.com/latex/templates/scu-undergraduate-thesis-template-unofficial/grwqfvgsfjxb}{\textcolor{blue}{undergraduate thesis template}}  修改的,同时追本溯源,来源应当是2013年学长的这个\href{https://github.com/dahakawang/scu_thesis_template}{\textcolor{blue}{repo}}。

目前模版是按照2025年计算机学院发送的本科生毕业论文设计模版修改的。很可能有遗漏且不符合《四川大学本科毕业论文(设计)管理办法(修订)-川大教【2022】56号》文件内具体要求的地方. 欢迎大家提PR(向这个\href{https://github.com/wodex1nhaoIeng/SCU_undergraduate_thesis_template}{\textcolor{blue}{repo}})。
\end{abstract}


\begin{abstractEng}{\LaTeX; Undergraduate; Thesis; Template}
LaTex thesis template is for SiChuan University students.The template is 
based on the original version, which was established by \textit{Flying of Death} in 2013. I have changed some places to fit in newly demand
(relevant document in 2018). Please note that it is an unofficial 
template. 
\end{abstractEng}

\tableofcontents

\clearpage

\pagenumbering{arabic}  % 重置页码并改为阿拉伯数字

	
	
	% 正文
	%--------------------------------------------------
	%	第一章
	\chapter{综述}
	
	
	%-----------------------------------------------
	\section{原版说明}
	大家好,本人系四川大学软件学院09级学生。由于个人习惯于在Linux下工作,一直以来便饱受没有word之不便。
	适逢大四毕业论文写作之时,本想使用\LaTeX{}\upcite{bobaru2009convergence} 来排版论文。但搜索一番,
	无奈发觉网上各类\LaTeX 模板虽然琳琅满目,却惟独没有我校本科毕业生所需之毕业设计论文模板。
	遂下决心自己制作一份本科生毕设模板,并将其共享出来,以方便大家。

	本模板格式参照教务处《四川大学本科毕业论文(设计)格式和参考文献著录要求》文件,同时以软件学院内部流传的一
	份word模板为蓝本制作而成,因此可能和别的学院毕设论文格式有所差异。如若有朋友发觉本模板板式和自己的学院不同,
	亦或者随着时间的推移本模板格式过时,你可以自行更改模板。当然,为了大家方便,请在github上新建一个branch,并
	发给我Pull Request。将你的劳动成果分享给大家,何乐而不为?
	
	
	%-------------------------------------------------------------
	\section{修订说明}
	
	\begin{enumerate}
		\item 本模板基于四川大学软件学院2009级前辈的模板上进行调整,同时参考了\textit{latex模板及使用说明0315-1.0}中的模板
		\item
		改动之处:页面边距、行间距、各级标题字体字号、各级标题占行、标题自动换行、附录添加、计算机代码格式设置、声明添加、致谢添加。以上改动按照2018版
		\textit{《四川大学本科毕业论文(设计)格式和参考文献著录要求》}
		\item 
		\LaTeX 突出的优点是模块化管理,虽然这份模板看上去文件比较多,但是容易上手。重点标注的红色文字是
		基本流程,只需按照\textcolor{red}{\nameref{procedure}}所述编辑相关文件即可
		\item
		在\textcolor{blue}{scuthesis.sty}中添加了一些注释,不建议初学者改动该文件,如有需要可以参考注释改动模板
		\item
		利用BibTex是规范且方便的。这里提供一个方法导出BibTex的参考文献格式:打开网站\textcolor{green}{\url{https://xs.dailyheadlines.cc/}},输入文献信息,找到相应的文献,点击下方的“引用”,打开对话框后点击底部的“BibTeX”即可导出
		\item
		该模板已上传至Overleaf,网址:\textcolor{green}{\url{https://overleaf.com/}}
		\item 
		最后推荐一个网站\textcolor{green}{\url{https://mathpix.com/}},可以导入图片、PDF等文件自动输出\LaTeX 代码
		
	\end{enumerate}
	
	
	
	%-------------------------------------------
	%	第二章
	\chapter{使用说明}
	
	
	%--------------------------------------------------------
	\section{环境设置}
	本模板需要依赖于\LaTeXe 、Xe\TeX 以及C\TeX ,因此在你使用前请确保这些发行版已经安装妥当。
	本文件全部使用UTF-8编码,并使用Xe\LaTeX 编译,以支持国际化和TrueType技术字体。


	%---------------------------------------------------------
	\section{文件结构}
	本模板由以下文件构成:
	\begin{itemize}
	\item \textcolor{blue}{main.tex} - \LaTeX 基本框架,你可以在此添加你需要的Package
	\item \textcolor{blue}{scuthesis.sty} - 川大毕设论文格式样式包,你不需要了解这个文件(除非本模板板式不合符你的需求)
	\item \textcolor{blue}{src/basic\_info.tex} - 定义论文作者基本信息
	\item \textcolor{blue}{src/prologue.tex} - 包含了封面、中英文摘要以及目录的定义
	\item \textcolor{blue}{src/chap*.tex} - 论文正文每章内容
	\item \textcolor{blue}{src/epilogue.tex} - 附录页
	\item \textcolor{blue}{ref/refs.bib} - 参考文献,bibtex文献库,推荐使用\href{http://jabref.sourceforge.net/}{JabRef} 维护
	\item \textcolor{blue}{src/declaration.tex} - 声明页
	\item \textcolor{blue}{src/acknowledgemen.tex} - 致谢页
	 
	\end{itemize}


	%-------------------------------------------------------
	\section{使用示例}
	首先说明一下,本教程是一篇self-contained的文章,本文章是直接编译本\LaTeX 模板得到,你可以具体参考模板源代码内容以学习如何使用。但是为了阐明脉络, 下面我将以一次完整使用的形式展示如何使用本模板。
	
	\subsection{编辑流程\label{procedure}}	% 三级标题
	\textcolor{red}{首先},填写自己的基本信息,例如姓名、学号之类,你需要打开basic\_info.tex文件将其填写进去。
	
	\textcolor{red}{其次},书写你的摘要,在prologue.tex里面是中英文摘要的定义处,你可以在其中编写摘要。假设你是\LaTeX 的老用户,你可能需要自己包含一些Package,那么你可以在main.tex中添加usepackage命令。
	
	\textcolor{red}{再者},在chap*.tex中编辑自己的正文,在epilogue.tex中编辑你的附录。
	
	\textcolor{red}{最后},在refs.bib中添加自己的参考文献,在acknowledgemen.tex中致谢。
	
	
	
	\subsubsection{四级标题}
	注意:2018年 \textit{《四川大学本科毕业论文(设计)格式和参考文献著录要求》}中目录标题不超过三级,
	所以四级标题不会在目录中显示。如需修改,在\textcolor{blue}{scuthesis.sty}中调整目录计数器。
	
	\subsection{数学环境}
	
	\begin{exmp}
		这是一个例子。
	\end{exmp}

	\begin{defn} \label{test}
		这是一个定义。
	\end{defn} 
	
	\begin{lem}
		引理
	\end{lem}
	
	\begin{thm}
		定理
	\end{thm}
	
	\begin{proof}
		Trivial
	\end{proof}
	
	\begin{cor}
		推论
	\end{cor}

	\begin{prop}
		命题
	\end{prop}
	
	
	%------------------------------------------------------
	\section{问题}
	\begin{enumerate}
	\item 中英文摘要分别不能超过一页,否则第二页的板式会有问题(由于本人精力有限,且该问题出现几率较小,目前暂未打算修复这个问题)。
	\item 所有文件必须是UTF-8编码,否则编译不能通过。
	\end{enumerate}

	
	
	% 后记(附录)
	%----------------------------------------------
%	附录Appendix
\appendix{
			\chapter{一维PD函数程序代码}
\begin{lstlisting}[language= Matlab]
%一维近场动力学函数, 二阶展开
A00 = [2 0 2/3; 0 2/3 0; 2/3 0 2/5];
b0 = [1 0 0]';
a0 = A00\b0;
xi = -1:0.001:1;
g0_2 = a0(1) + a0(2) * xi + a0(3) * xi.^2;

\end{lstlisting}
}
	
	
	% 参考文献
	\cleardoublepage
	\addcontentsline{toc}{chapter}{参考文献}		% 在目录中添加参考文献
	\bibliographystyle{unsrt}
	\bibliography{ref/refs}
	
	
	% 声明
	\clearpage
	\addcontentsline{toc}{chapter}{声\hspace{0.8cm}明}
	%-----------------------------------------------------------------------
%declaration.tex

	\chapter*{声\hspace{0.8cm}明}


%-------------------------------------------------------------------
	\songti \zihao{5}本人声明所呈交的学位论文是本人在导师指导下进行的研究工作及取得的研究成果。据我所知,除了文中特别加以标注和致谢的地方外,论文中不包含其他人已经发表或撰写过的研究成果,也不包含为获得四川大学或其他教育机构的学位或证书而使用过的材料。与我一同工作的同志对本研究所做的任何贡献均已在论文中作了明确的说明并表示谢意。
	
	本学位论文成果是本人在四川大学读书期间在导师指导下取得的,论文成果归四川大学所有,特此声明。
	
	\vspace{15pt}
		% \begin{tabular}{b{4cm} >{\centering\arraybackslash}b{2.5cm} }
	\songti \zihao{5} 作者签名: \hspace{4cm} \songti \zihao{5} 导师签名:
		% \end{tabular}
        \vspace{15pt}
        \begin{center}
            年 \hspace{0.5cm} 月 \hspace{0.5cm} 日
        \end{center}

        \vspace{140pt}

        \begin{center}
            \heiti \zihao{-3} {学位论文使用授权书}
        \end{center}

        \songti \zihao{5}本学位论文作者完全了解四川大学有关保留、使用学位论文的规定,同意学校保留并向国家有关部门或相关机构送交论文的原件、复印件和电子版,允许论文被查阅和借阅。本人授权四川大学将本学位论文的全部或部分内容编入有关数据库进行信息技术服务,可以采用影印、缩印或扫描等复制手段保存、汇编学位论文,并用于学术活动。
	
	(涉密学位论文在解密后适用于本授权书)
	
	\vspace{15pt}
		% \begin{tabular}{b{4cm} >{\centering\arraybackslash}b{2.5cm} }
	\songti \zihao{5} 作者签名: \hspace{4cm} \songti \zihao{5} 导师签名:
		% \end{tabular}
        \vspace{15pt}
        \begin{center}
            年 \hspace{0.5cm} 月 \hspace{0.5cm} 日
        \end{center}
	

	% 致谢
	\clearpage
	\addcontentsline{toc}{chapter}{致\hspace{0.8cm}谢}
	%--------------------------------------------------
%	致谢 Acknowlegement
	\chapter*{致\hspace{0.8cm}谢}
	Over the course of my researching and writing this paper, I would like to express my thanks to all those who have helped me.

	
	
\end{document} 
	